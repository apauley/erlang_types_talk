\documentclass{beamer}

\usetheme{Warsaw}
%% \usetheme{EastLansing}
%% \usecolortheme{beetle}

\title[Types in Erlang]{Optional Type Specification in Erlang}
\author{Andreas Pauley -- @apauley}
\institute{Lambda Luminaries -- @lambdaluminary}
\date{November 11, 2013}

\usepackage[utf8]{inputenc}
\usepackage{graphicx}

\usepackage{minted}

\usepackage{hyperref}

\AtBeginSection[]
{
  \begin{frame}
    \frametitle{Table of Contents}
    \tableofcontents[currentsection]
  \end{frame}
}

\begin{document}

\begin{frame}
  \titlepage
\end{frame}

\section{Erlang Type Overview}

\begin{frame}{Dynamic, but strong}

  \inputminted[firstline=5]{erlang}{src/dynamic.erl}

\end{frame}

\begin{frame}[fragile]{Dynamic, but strong}

  \begin{minted}{erlang}
3> 5 + "6".
** exception error: an error occurred when evaluating
    an arithmetic expression
     in operator  +/2
        called as 5 + "6"
  \end{minted}

\end{frame}

\section{Can we add compile-time types to Erlang?}

\begin{frame}{Yes and No :-/}

  \begin{itemize}[<+->]
  \item The Haskell giants Simon Marlow and Philip Wadler spent over a year
    developing a static type system for Erlang (circa 1997).
  \item Only a subset of the language could be type-checked.
  \item Process types and inter-process messages could not be type-checked.
  \end{itemize}

\end{frame}

\section{Optional Types}

\begin{frame}{Declaring Types}

We can declare union/sum types and product types:

  \inputminted[firstline=4, lastline=6]{erlang}{src/cards.erl}

\end{frame}

\begin{frame}{Declaring Types}

Types in record definitions:

  \inputminted[firstline=10]{erlang}{src/rsa_id_number.hrl}

\end{frame}


\begin{frame}{Using Types}

  \inputminted[firstline=8, lastline=9]{erlang}{src/cards.erl}

\end{frame}


\begin{frame}[fragile]{Using Types}

  \begin{minted}{erlang}
-type gender() :: male | female.
-type rsa_id_number() :: #rsa_id_number{}.
-export_type([rsa_id_number/0]).

-spec gender(rsa_id_number()) -> gender().
gender(#rsa_id_number{gender_digit=GenderDigit})
  when GenderDigit >= 5 -> male;
gender(#rsa_id_number{gender_digit=GenderDigit})
  when GenderDigit < 5  -> female.
  \end{minted}

\end{frame}


\section{Dialyzer}


\begin{frame}{Discrepency Analyzer for Erlang}

  \begin{itemize}[<+->]
  \item A static analysis tool developed by Kostis Sagonas from the
    University of Uppsala in Sweden.
  \item Type inference based on Success Typing instead of
    Hindley-Milner.
  \item Doesn't need type specifications, but you get more useful
    results if you provide them.
  \end{itemize}

\end{frame}


\begin{frame}{Dialyzer and Types}

  \inputminted[firstline=4, lastline=6]{erlang}{src/cards.erl}

\end{frame}

\begin{frame}{Dialyzer and Types}

  \inputminted[firstline=11]{erlang}{src/cards.erl}

\end{frame}

\begin{frame}[fragile]{Dialyzer and Types}

  \begin{minted}{erlang}
  Checking whether the PLT .deps_plt is up-to-date... yes
  Proceeding with analysis...
src/cards.erl:15: Function main/0 has no local return
src/cards.erl:18: The call cards:kind({'rubies',4})
    breaks the contract (card()) -> 'face' | 'number'
 done in 0m0.54s
done (warnings were emitted)
make: *** [dialyzer] Error 2
  \end{minted}

\end{frame}

\section{Proper}


\begin{frame}{Discrepency Analyzer for Erlang}

  \begin{itemize}[<+->]
  \item A QuickCheck-inspired property-based testing tool for Erlang.
  \item Developed by Manolis Papadakis, Eirini Arvaniti and Kostis
    Sagonas from the National Technical University of Athens.
  \item Has the ability to generate random tests based on type specs.
  \end{itemize}

\end{frame}

\begin{frame}[fragile]{Property tests from the specs}

  \begin{minted}{erlang}
2> proper:check_specs(rsa_id_number).
Testing rsa_id_number:checksum/1
....................................................................................................
OK: Passed 100 test(s).

Testing rsa_id_number:citizen/1
....................................................................................................
OK: Passed 100 test(s).

Testing rsa_id_number:gender/1
....................................................................................................
OK: Passed 100 test(s).
  \end{minted}
\end{frame}

\end{document}

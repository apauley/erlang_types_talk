\documentclass{beamer}

\usetheme{Warsaw}
%% \usetheme{EastLansing}
%% \usecolortheme{beetle}

\title[Types in Erlang]{Optional Type Specification in Erlang}
\author{Andreas Pauley -- @apauley}
\institute{Lambda Luminaries -- @lambdaluminary}
\date{November 11, 2013}

\usepackage[utf8]{inputenc}
\usepackage{graphicx}

\usepackage{minted}

\usepackage{hyperref}

\AtBeginSection[]
{
  \begin{frame}
    \frametitle{Table of Contents}
    \tableofcontents[currentsection]
  \end{frame}
}

\begin{document}

\begin{frame}
  \titlepage
\end{frame}

\section{Erlang Type Overview}

\begin{frame}{Dynamic, but strong}

  \inputminted[firstline=5]{erlang}{src/dynamic.erl}

\end{frame}

\begin{frame}[fragile]{Dynamic, but strong}

  \begin{minted}{erlang}
3> 5 + "6".
** exception error: an error occurred when evaluating
    an arithmetic expression
     in operator  +/2
        called as 5 + "6"
  \end{minted}

\end{frame}

\section{Can we add compile-time types to Erlang?}

\begin{frame}{Yes and No :-/}

  \begin{itemize}[<+->]
  \item The Haskell giants Simon Marlow and Philip Wadler spent over a year
    developing a static type system for Erlang (circa 1997).
  \item Only a subset of the language could be type-checked.
  \item Process types and inter-process messages could not be type-checked.
  \end{itemize}

\end{frame}

\section{Optional Types}

\begin{frame}{Declaring Types}

  \inputminted[firstline=4, lastline=6]{erlang}{src/cards.erl}

\end{frame}


\begin{frame}{Using Types}

  \inputminted[firstline=12, lastline=13]{erlang}{src/cards.erl}

\end{frame}


\section{Dialyzer}


\begin{frame}{Discrepency Analyzer for Erlang}

  \begin{itemize}[<+->]
  \item A static analysis tool developed by Kostis Sagonas from the
    University of Uppsala in Sweden.
  \item Type inference based on Success Typing instead of
    Hindley-Milner.
  \item Doesn't need type specifications, but you get more useful
    results if you provide them.
  \end{itemize}

\end{frame}


\section{Proper}

\end{document}
